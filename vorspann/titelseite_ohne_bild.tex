%
% Project documentation template
% ===========================================================================
% This is part of the document "Project documentation template".
% Authors: brd3, kaa1
%

\begin{titlepage}


% BFH-Logo absolute placed at (28,12) on A4 
% Actually not a realy satisfactory solution but working.
%---------------------------------------------------------------------------
\setlength{\unitlength}{1mm}
\begin{textblock}{20}[0,0](28,12)
	\includegraphics[scale=1.0]{bilder/BFH_Logo_B.png}
\end{textblock}
\color{black}

% Institution / Titel / Untertitel / Autoren / Experten:
%---------------------------------------------------------------------------
\begin{flushleft}

\vspace*{21mm}

\fontsize{26pt}{40pt}\selectfont 
\titel 				\\							% Titel aus der Datei vorspann/titel.tex lesen
\vspace{2mm}

\fontsize{16pt}{24pt}\selectfont\vspace{0.3em}
Hier steht ein Untertitel 			\\							% Untertitel eingeben
\vspace{5mm}

\fontsize{10pt}{12pt}\selectfont
\textbf{Art der Arbeit (Semesterarbeit / Bachelorthesis / etc.)} \\									% eingeben
\vspace{7mm}

% Abstract (eingeben):
%---------------------------------------------------------------------------
\begin{textblock}{150}(28,100)
\fontsize{10pt}{12pt}\selectfont
[Kurztext (Abstract) einf�gen, falls gew�nscht] \\ 
Dieses Dokument dient als Vorlage f�r die Erstellung von Berichten nach den Richtlinien der BFH. Die Vorlage ist in \LaTeX{} erstellt und unterst�tzt das automatische Erstellen von diversen Verzeichnissen, Literaturangaben, Indexierung und Glossaren. Dieser kleine Text ist eine Zusammenfassung �ber das vorliegenden Dokument mit einer L�nge von 4 bis max. 8 Zeilen. \\
Das Titelbild kann in den Zeilen 157/158 der Datei template.tex ein- oder ausgeschaltet werden.
\end{textblock}

\begin{textblock}{150}(28,225)
\fontsize{10pt}{17pt}\selectfont
\begin{tabbing}
xxxxxxxxxxxxxxx\=xxxxxxxxxxxxxxxxxxxxxxxxxxxxxxxxxxxxxxxxxxxxxxx \kill
Studiengang:	\> [z.B. Elektro- und Kommunikationstechnik]	\\			% Namen eingeben
Autoren:		\> [Test Peter, M�ster R�s�]		\\					% Namen eingeben
Betreuer:	\> [Dr.~Xxxx Xxxx, Dr.~Yyyy Yyyy]		\\					% Namen eingeben
Auftraggeber:	\> [Wwwww AG]						\\					% Namen eingeben
Experten:		\> [Dr.~Zzzz Zzzz]				\\					% Namen eingeben
Datum:			\> \versiondate					\\		% aus Datei vorspann/version.tex lesen
\end{tabbing}

\end{textblock}
\end{flushleft}

\begin{textblock}{150}(28,280)
\noindent 
\color{bfhgrey}\fontsize{9pt}{10pt}\selectfont
Berner Fachhochschule | Haute �cole sp�cialis�e bernoise | Bern University of Applied Sciences
\color{black}\selectfont
\end{textblock}


\end{titlepage}

%
% ===========================================================================
% EOF
%
